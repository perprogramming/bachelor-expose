\section{Inhalt der Arbeit}

\subsection{Problemstellung}

Diese Bachelorarbeit entsteht im Rahmen der Tätigkeit als Softwareentwickler bei der Pixelhouse GmbH. Diese betreibt seit ca. 10 Jahren die Webseite Chefkoch.de, Europas größtes Kochportal. Neben der Vermarktung der Webseite und der Betreuung der Community wird von den Mitarbeitern vor allem auch die softwaretechnische Entwicklung der Plattform und das Hosting durchgeführt.

Die Webseite Chefkoch.de basiert auf einer seit Beginn des Projektes stetig wachsenden PHP-Anwendung. Große Teile der Anwendung sind dabei ohne fundierte Kenntnisse in der Softwareentwicklung entstanden. Dies merkt man vor allem an einer fehlenden Systemarchitektur und einer unübersichtlichen und sehr redundanten Struktur der Daten.

Auch die auf Linux/Ubuntu-Systemen basierende Infrastruktur der Chefkoch-Anwendung ist mit den Jahren sehr komplex geworden ist, vor allem auch, um die hohe Last von mehreren Millionen Besuchern täglich bedienen zu können. So werden Loadbalancer, mehrere Application-Server und diverse Persistenz-Dienste wie SQL-Datenbanken und Key-Value-Stores eingesetzt.

Der Wartungsaufwand und auch die benötigte Zeit für Neuentwicklungen sind entsprechend hoch. Die Herausforderung für das aktuelle Entwicklungsteam besteht darin, neben dem Alltagsgeschäft und der Umsetzung neuer Produktideen auch grundlegende Modernisierungen am System vorzunehmen.

Hierbei ist es Vorgabe des Managements, bestehende und neue Funktionen über automatisierte Tests abzusichern. Gerade bei bestehenden Funktionen kommen hierbei oft nur Systemtests in Frage, da die zugrundeliegende Software wenig modular aufgebaut ist und so kaum Unit- oder Integrationstests erlaubt.

Diese Systemtests werden mit Hilfe des Testtools Behat implementiert, das die Webdriver-Schnittstelle dazu verwendet, echte Webbrowsern über die Seite laufen zu lassen, um die benötigten Funktionen abzutesten.

Das größte Problem mit dieser Art von Systemtest ist, dass sie vergleichsweise langsam sind und bereits heute mehrere Stunden laufen. Dies bedeutet, dass man nach fertiger Implementierung einer neuen Funktion oder der Modernisierung einer bestehenden Komponente sehr lange auf das Testergebnis warten muss und so Zeit vergeht, bis sich die Änderung in den Hauptentzwicklungszweig integrieren oder sogar in Produktion nehmen lässt.

Eine mögliche Lösung hierfür wäre es, mehrere Testumgebungen anzubieten, um das Ausführen der automatischen Systemtests parallelisieren zu können oder auch einfach manuelle Abnahmen durch das Produktmanagement zu unterstützen.

Stand heute ist es allerdings sehr aufwändig, neue Testumgebungen aufzusetzen. Der entsprechende Prozess kann mitunter mehrere Tage dauern, da Softwareentwicklung und Betrieb hier wenig effizient zusammenarbeiten und viele Teilschritte manuell erfolgen.

Die Testumgebungen sind außerdem meist wenig isoliert und laufen z.B. auf den gleichen Datenbankservern oder hinter dem gleichen Loadbalancer. So lassen sich Änderungen an diesen Infrasktruktur-Komponenten heute mitunter gar nicht testen.

\subsection{Fragestellung}

Virtualisierungstechniken im Allgemeinen erlauben die einfache Replikation von Systemumgebungen, sind also ein möglicher Kandidat, um solche Problemstellungen auf technischer Ebene zu lösen. Dabei gibt es unterschiedliche Ansätze der Virtualisierung, die sich in ihrer Funktionalität, aber auch in ihrer Leistung und ihrem Verwaltungsaufwand unterscheiden. Andere Lösungen, wie z.B. Änderungen an der Organisationsstruktur, dem Entwicklungsprozess, dem eigentlichen Programmcode oder der Infrasktruktur selbst sind nicht Teil der Fragestellung.

\subsection{Zielsetzung}

Ziel dieser Bachelorarbeit ist es zu untersuchen, ob man mit Hilfe von Virtualisierungstechniken eine Lösung für das oben beschrieben Problem finden kann. Mit dieser Lösung soll nicht nur die Anwendung sondern auch deren Infrastruktur in einem konkreten Zustand nachgehalten und effizient aufgesetzt werden können, um so einfach und beliebig oft Testumgebungen anbieten zu können. Neben der konzeptionellen Arbeit soll auch eine prototypische Umsetzung und nachfolgende Bewertung erfolgen.

\subsection{Theoriebezug}

Der hohe Zeitaufwand für das Aufsetzen neuer Testumgebungen und das Durchführen der Systemtests widersprechen schon lange etablierten Softwareentwicklungsmethodiken wie Continuous Integration und Continuous Delivery. Diese zielen darauf ab, kurze Feedbackschleifen für Entwickler und Produktmanager zu ermöglichen, in dem sich neue Versionen der Software kurzfristig mit denen anderer Entwickler integriert lassen und ebenso kurzfristig in Produktion genommen werden können, um so auch Feedback von echten Benutzern zu erhalten.

Die Tatsache, dass sich bestimmte Infrastrukturkomponenten gar nicht testen lassen und dass die Zusammenarbeit zwischen Entwicklern und Betrieb hier wenig effizient erfolgt, werden von der Wissenschaft heute bereits mit der DevOps Bewegung beantwortet.

Wie konkrete Virtualisierungsansätze, z.B. die Verwendung von Containern, hier effizient Abhilfe schaffen kann, ist dabei allerdings weniger umfangreich beleuchtet.

\subsection{Vorgehensweise}

Diese Bachelorarbeit soll zunächst einen Überblick über relevante Softwareentwicklungsmethodiken und die für diese Problemstellung relevanten Virtualisierungsansätze geben und diese in Hinsicht auf die Problemstellung bewerten. Basierend auf zwei ausgesuchten Ansätzen soll jeweils ein Konzept zur Realisierung für die konkrete Problemstellung angegeben und prototypisch umgesetzt werden. Nachfolgend sollen die Implementierungen evaluiert werden.

\subsection{Evaluationsstrategie}

Anhand der Problemstellung lassen sich Kriterien finden und gewichten, mit deren Hilfe sich die prototypisch umgesetzten Lösungen bewerten lassen. Ein wichtiges Kriterium ist auf jeden Fall die Reduzierung der Gesamtausführungsdauer der Tests aber z.B. auch die einfache Verwendbarkeit der Lösung. Andere Kriterien wie z.B. monetäre Kosten oder die Virtualisierbarkeit anderer Umgebungen (Windows, MacOS) fallen hingegen weniger ins Gewicht.