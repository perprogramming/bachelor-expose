\section{Inhalt der Arbeit}

\subsection{Problemstellung}

Die Webseite Chefkoch.de basiert auf einer PHP-Anwendung, deren Entwicklung vor über 10 Jahren begann. Große Teile der Anwendung sind dabei ohne fundierte Kenntnisse in der Softwareentwicklung entstanden. Dies merkt man vor allem an einer fehlenden Systemarchitektur und einer unübersichtlichen und sehr redundanten Struktur der Daten. Der Wartungsaufwand und auch die benötigte Zeit für Neuentwicklungen sind entsprechend hoch. Die Herausforderung für das aktuelle Entwicklungsteam besteht also darin, neben dem Alltagsgeschäft auch grundlegende Moderniesierungen am System vorzunehmen. Da die Anwendung bislang auch keine Softwaretests zur Qualitätssicherung besitzt, ist es vor größeren Umbauarbeiten zunächst notwendig, die entsprechenden Bausteine über Funktionstests abzusichern. Auch hier fehlt es an einer entsprechenden Infrastruktur, vor allem auch im Bereich des Testdatenmanagements.

\subsection{Zielsetzung}

Ziel dieser Bachelor-Arbeit ist es, den Entwurf und die Implementierung von Werkzeugen zu erarbeiten, die aufgrund des aktuellen Stands der Forschung für das Testdatenmanagement der automatischen Funktionstests bei Chefkoch.de sinnvoll und notwendig sind.

\subsection{Vorgehensweise}

Nach der Einordnung automatischer Funktionstests als Teil üblicher Qualitätssicherungsmaßnahmen im Bereich des Softwareengineerings, wird zunächst die besondere Bedeutung des Testdatenmanagements betrachtet und verschiedene Strategien analysiert, die hierbei angewendet werden können. Hierbei geht es zum einen um das sogenannte Datensubsetting und zum anderen um die synthetische Testdatengenerierung. Anschließend werden konkrete Werkzeuge für die Integration des Testdatenmanagements in den Entwicklungsprozess bei Chefkoch.de entworfen und implementiert.

\subsection{Unternehmensportrait}

Die Pixelhouse GmbH ist der Betreiber von Chefkoch.de, Europas größtem Kochportal. Neben der Vermarktung der Webseite und der Betreuung der Community mit rund 1,5 Millionen Benutzern wird vor allem auch die softwaretechnische Entwicklung der Plattform und deren Betrieb durch die Pixelhouse GmbH durchgeführt.

Das Unternehmen wurde im Jahre 1998 als klassische Internetagentur gegründet. Nachdem sich der Betrieb der Seite Chefkoch.de als großer Erfolg erwies, ist sie heute das einzige Produkt des Unternehmens. Im Jahre 2008 wurde die Pixelhouse GmbH von Gruner + Jahr übernommen und ist inzwischen eine 100\%-tige Tochter des renommierten Medienhauses.

Das Unternehmen beschäftigt ca. 40 Mitarbeiter in zwei Standorten in Bonn und Köln. Der Standort Köln betreibt vor allem die Entwicklung der mobilen Apps (iOS, Android) für Chefkoch.de, alle anderen Abteilungen befinden sich im Standort Bonn.
